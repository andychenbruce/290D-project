\documentclass{article}
\usepackage[margin=1.0in]{geometry}
\usepackage{hyperref}
\usepackage{amsmath}
\usepackage{amsfonts}
\title{Differential Programming for Automatic Program Correction}
\author{Andrew Bruce \\ \href{mailto:acbruce@ucsc.edu}{acbruce@ucsc.edu}
  \and Dongjing Wang \\ \href{mailto:dwang114@ucsc.edu}{dwang114@ucsc.edu}
  \and Ming Qi \\ \href{mailto:mqi6@ucsc.edu}{mqi6@ucsc.edu} }

\begin{document}
\maketitle
\section*{Motiviation}
Approximating gradient descent on discreet code. When SWE write tests for their code it can act as a loss function.

\section*{Theory and Implementation}
This draft of a textbook \cite{blondel2024elementsdifferentiableprogramming} released during summer 2024 provides the background for approximating program data structure operations and control flow as a differentiable functions like a neural network.\\
An example is if $\vec{l} \in \mathbb{R}^{n_{\ge 1}}$ is a list and there is some discrete function such as $max$:
\begin{center}
  $\mathrm{max} \in \mathbb{R}^{n_{\ge 1}} \rightarrow \mathbb{R}$\\
  $\mathrm{max}(\vec{l}) = l_{\mathrm{max}}$
\end{center}
But when attempting to differentiate
\begin{center}
  $$\dfrac{d \mathrm{max}(\vec{l})}{dl_i} = \left\{
  \begin{array}{ll}
    1 & l_i = l_{\mathrm{max}} \\
    0 & \mathrm{otherwise}
  \end{array} 
  \right.$$
\end{center}
Giving null derivatives when changing components of the list that are not the current maximum element.
One can define a ``soft'' approximation of this discrete functions from a stoichastic point of view, usually having the advantage of differentiability.
\begin{center}
  $\mathrm{max} \approx \mathrm{softmax}$\\
  $\mathrm{softmax} \in \mathbb{R}^n \rightarrow \mathbb{R}^n$\\
  $\mathrm{softmax}(\vec{l})_i = \dfrac{e^{l_i}}{\sum_{j=1}^n e^{\vec{l}_j}}$\\
\end{center}
This can be generalized to other programming concepts such as logical comparisons, data structures, and control flow.
\begin{center}
  \begin{tabular}{ |c|c| }
    \hline
    normal & soft variant \\
    \hline
    $(\ge) = \mathrm{ge} \in (\mathbb{R} \times \mathbb{R}) \rightarrow \{ 0, 1 \}$ & $\mathrm{softge} \in (\mathbb{R} \times \mathbb{R}) \rightarrow [ 0, 1 ]$\\
    $(=) = \mathrm{eq} \in (\mathbb{R} \times \mathbb{R}) \rightarrow \{ 0, 1 \}$ & $\mathrm{softeq} \in (\mathbb{R} \times \mathbb{R}) \rightarrow [ 0, 1 ]$\\
    $\mathrm{ifelse} \in (\{ 0, 1 \} \times F \times F) \rightarrow F$ & $\mathrm{softifelse} \in ([ 0, 1 ] \times F \times F) \rightarrow F$\\
    $\mathrm{listget} \in (\mathbb{R}^n \times [n] ) \rightarrow \mathbb{R}$ & $\mathrm{softlistget} \in (\mathbb{R}^n \times \Delta^n ) \rightarrow \mathbb{R}$\\
    $\mathrm{listset} \in (\mathbb{R}^n \times \mathbb{R} \times [n] ) \rightarrow \mathbb{R}^n$ & $\mathrm{softlistset} \in (\mathbb{R}^n \times \mathbb{R} \times \Delta^n ) \rightarrow \mathbb{R}^n$\\
    $\mathrm{listinsert} \in (\mathbb{R}^n \times \mathbb{R} \times [n+1] ) \rightarrow \mathbb{R}^{n+1}$ & $\mathrm{softlistinsert} \in (\mathbb{R}^n \times \mathbb{R} \times \Delta^{n+1} ) \rightarrow \mathbb{R}^{n+1}$\\
    \hline
  \end{tabular}
  
\end{center}
We would have to manually write a the gradient calculation, back propigation and newtons method. Possibly by abusing numpy autodiff.
\section*{Example}

\begin{verbatim}
void merge_sort(array list, int len){
    if((len == 0) || (len == 1)){
        return;
    }
    if(len == 2){
        if(list.get(0) > list.get(1)){
            int tmp = list.get(0);
            list.set(0, list.get(1));
            list.set(1, tmp);
        }
        return;
    }
    int split_point = len / 2;
    merge_sort(list, split_point+1);
    merge_sort(list+split_point, len-split_point);
    merge(list, split_point);
}
\end{verbatim}
In the above example the code is incorrect because of the first call to \verb|merge_sort| having an index of \verb|split_point+1| rather than \verb|split_point|, an off by one error. After transforming the algorithm into a differentiable function, \verb|merge_sort| $\longrightarrow$ \verb|soft_merge_sort|, the index \verb|split_point+1| can be transformed into \verb|split_point+C|, with the ``weight'' $C = 1$ as a starting value. Then using back propigation to calculate $\nabla L($\verb|soft_merge_sort|$)$, where $L$ is a loss function based of some test cases, should in theory it should be able to correct the off by one error by changing $C$ from $1$ to $0$ through a simple gradient descent algorithm.
\section*{Previous work}
One such previous work is ``Zygote'' \cite{DBLP:journals/corr/abs-1907-07587}, a framework for differentiating Julia programs. Another example includes a purely function IR language called ``Myia'' \cite{DBLP:journals/corr/abs-1810-11530}. Does not support list or dictionary operations such as getting and setting. Is an exact transform of the program souce code to a purely functional IR language.\\
Both of these seem to be an exact transformation rather than an approximation meaning that it won't work for a lot of discreet operations or discontinous functions causing null derivatives as part of the gradient preventing gradient descent from optimizing that parameter any further.
These both are also used to back propigate from the inputs of the program, but not differentiate with respect to the program parameters itself. They have not been applied with back propigation for for automatic code correction.

\section*{Novelty}
As far as I know no one has done this before.

\section*{Data collection}
The general algorithm doesn't need data, but when running it on a program it will create an instance of a learning problem that would have data in the form of test cases. We will have to manually write programs and test cases for each program to see if the algorithm converges on the correct answer.

\section*{Limitation}
Have not verified that you can take the second derivative, the Hessian, for the second order Newtons method $x_{\mathrm{new}} = x_{\mathrm{old}} - [\nabla^2 f]^{-1}(x) \nabla f(x)$.\\
The gradient is applied on the approximation of a program, not the actual program. When the loss of the approximation converges to zero, it does not necessarily mean that the real program will become correct.
\bibliographystyle{abbrv}
\bibliography{refs}
\end{document}
